\section{Tablero}

Pasando al game loop del juego primero debemos manejar el entorno donde vamos a jugar asi que primero dibujaremos el entorno
Para ello lo primero que se hice es que pasamos al modo 13h para poder usar colores, lo siguiente es dibujar nuestro tablero
para esto dibuje un rectangulo en medio de la pantalla y luego dibuje los bordes de bloques de 1x1, en este caso sera un 
tablero de 10x16, con cuadros de 12x12 pixeles

La logica para dibujar los rectangulos y bordes son simples loops definidos en este codigo

\begin{lstlisting}[style=estiloASM,caption={Funcion para el dibujar el tablero}]

DibujarTablero PROC
    MOV AH, 0Ch
    MOV AL, background_colour
    MOV DX, play_ground_start_row
.LOOP1:
    MOV CX, play_ground_start_col
.LOOP2:
    INT 10h
    INC CX
    CMP CX, play_ground_finish_col
    JNZ .LOOP2
    INC DX
    CMP DX, play_ground_finish_row
    JNZ .LOOP1
    RET
DibujarTablero ENDP

\end{lstlisting}

\begin{lstlisting}[style=estiloASM,caption={Funcion para dibujar los bordes}]
DibujarBorde PROC
    MOV BX, play_ground_start_row
    MOV block_start_row, BX
    ADD BX, 12
    MOV block_finish_row, BX
.Dibujar:
    MOV BX, play_ground_start_col
    MOV block_start_col, BX
    ADD BX, 12
    MOV block_finish_col, BX
    .Dibujar2:
        CALL Dibujar_Borde_Bloque_Unico
        ADD block_start_col, 12
        ADD block_finish_col, 12
        MOV BX, play_ground_finish_col
        CMP BX, block_start_col
        JNZ .Dibujar2
    ADD block_start_row, 12
    ADD block_finish_row, 12
    MOV BX, play_ground_finish_row
    CMP BX, block_start_row
    JNZ .Dibujar
    RET
DibujarBorde ENDP
\end{lstlisting}

Como resultado de esto podemos obervar en pantalla lo siguiente

\begin{figure}[h!]
	\begin{center}
		\includegraphics[height = 5cm]{./figures/TableroVacio.png}
	\end{center}
	\caption{Tablero inicial}\label{fig:TableroInicial}
\end{figure}

