\section{Siguiente pieza}

Para terminar con la parte de nuestro entorno inicial necesitamos 
saber la cola de las piezas siguientes, para ello lo explicare a 
grandes rasgos lo que hago

\begin{enumerate}
	\item Generar dos numeros aleatorios entre 0 a 4 y dibujarlos
	\item Sacar el primer numero generado para que sea la primer pieza a caer y dibujarla en el tablero
	\item Generar aleatoriamente otra pieza (numero de 0 a 4)
	\item Actualizar las piezas siguientes
\end{enumerate}

Para esto ya se implementan funciones para dibujar las piezas en nuestro tablero,
se ocupan funciones que generen numeros aleatorios.
Ahora surgue un problema que pasa si tenemos un estado donde ya no podemos seguir metiendo piezas, es decir, 
hemos perdido, pues lo que se hace es poner un validador de que la pieza a dibujar no tiene algo que le estorbe,
de ser asi se dibuja en otra posicion y si ya no tiene mas espacio para futuro el juego termina.
Asi que esto ultimo solo requiere la nueva funcion que valide si la pieza siguiente puede ser dibujada, y si es 
asi con instrucciones de JE y JMP se eligue la figura para dibujar, el resultado queda de la siguiente manera

\begin{figure}[h!]
	\includegraphics[height = 5cm]{./figures/TableroFinal.png}
	\caption{Tablero Final Dibujado}\label{fig:TF}
\end{figure}

