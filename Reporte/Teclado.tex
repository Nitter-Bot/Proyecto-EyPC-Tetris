\section{Teclado}

Ahora por como sirve tetris la pieza empezara a caer cada cierto intervalo de tiempo, por lo tanto ya podemos a 
empezar a jugar, por lo que necesitamos saber como decirle al programa que queremos hacer, si movernos a la
izquierda, a la derecha , bajar un poco, rotar la pieza o tirar la pieza

Lo que hacemos es leer una tecla, si es alguna de las que podemos usar para movernos hacemos la accion que corresponde
\begin{itemize}
	\item W - Nos permite rotar la pieza
	\item A - Nos movemos a la izquierda
	\item S - Bajamos la pieza una casilla
	\item D - Nos movemos a la derecha
	\item F - Tirar la pieza
\end{itemize}

\begin{lstlisting}[style=estiloASM,caption={Implementacion del teclado}]
LlamarTeclado PROC
    MOV delay_counter, 1
delay_loop3:
    MOV CX, 0FFFFH
    INC delay_counter
delay_loop4:
    CALL Acciones
    LOOP delay_loop4
    CMP delay_counter, 5
    JNZ delay_loop3
    RET
LlamarTeclado ENDP

Acciones PROC
    MOV AH, 01h
    INT 16h
    JZ arrow_exit

    MOV AH, 00h
    INT 16h

    CMP AL, 'd'
    JE shape_shift_right_label
    CMP AL, 'a'
    JE shape_shift_left_label
    CMP AL, 's' 
    JE shape_shift_down_label 
    CMP AL, 'w'
    JE shape_rotate_label
    CMP AL, 'f'
    JE fast_shape_shift_down_label

    JMP arrow_exit

shape_shift_right_label: 
    CALL shape_shift_right
    JMP arrow_exit

shape_shift_left_label:
    CALL shape_shift_left
    JMP arrow_exit 

shape_shift_down_label:
    CALL shape_shift_down
    JMP arrow_exit

shape_rotate_label:
    CALL shape_rotate
    JMP arrow_exit 

fast_shape_shift_down_label:
fast_loop:
    CALL shape_shift_down
    CMP successful_magic_shift, 0H
    JE arrow_exit
    JMP fast_loop

arrow_exit:
    RET
Acciones ENDP

\end{lstlisting}


