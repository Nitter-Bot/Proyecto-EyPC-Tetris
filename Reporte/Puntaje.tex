\section{Puntaje}

Luego dibujamos nuestro sistema de puntaje que llevamos para ello lo que se va a hacer es
simplemente en una posision marcada poner el puntaje , y luego podemos usar nuestra funcion para actualizar el score, que lo que hace es una conversion 
binaria a ASCII para actualizar y mostrar nuestro puntaje, ambas funciones se implementaron de la siguiente forma
y se ve de la siguiente forma

\begin{lstlisting}[style=estiloASM,caption={Funciones para mostrar y actualizar el puntaje}]
DisplayScore PROC
    MOV AH, 02h
    MOV BH, 00h
    MOV DH, 04h
    MOV DL, 01h
    INT 10h
    MOV AH, 09h
    LEA DX, msg_score
    INT 21h
    RET
DisplayScore ENDP

UpdateScore PROC
    XOR AX, AX
    MOV SI, 9
    MOV AX, score
    MOV BX, 10
.label:
    CMP SI, 5
    JE .exit_label
    XOR DX, DX
    DIV BX
    ADD DX, 30h
    MOV [msg_score+si], DL
    DEC SI
    JMP .label
.exit_label:
    CALL DisplayScore
    RET
UpdateScore ENDP
\end{lstlisting}

Se ve en nuestro entorno asi 

\begin{figure}[h!]
	\begin{center}
		\includegraphics[height = 5cm]{./figures/score.png}
	\end{center}
	\caption{Score dibujado}\label{fig:score}
\end{figure}

